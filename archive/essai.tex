\documentclass[french]{beamer}

\usepackage[utf8]{inputenc}
\usepackage[T1]{fontenc}
\usepackage{lmodern}
\usepackage{amsmath, amssymb}

\usepackage{babel}


%CHOIX DU THEME et/ou DE SA COULEUR
% => essayer différents thèmes (en décommantant une des trois lignes suivantes)
\usetheme{PaloAlto}
%\usetheme{Madrid}
%\usetheme{Copenhagen}

% => il est possible, pour un thème donné, de modifier seulement la couleur


%\useoutertheme[left]{sidebar}


%Pour le TITLEPAGE
\title{Exemple de Beamer}
\subtitle{Initiation master~1}
\author[Nom (court)]{Nom (long) de l'auteur}
\date{Janvier 2012}
\institute[UT3 -- FSI]{Université Toulouse~3 -- Faculté des sciences et ingénierie}


\begin{document}

\begin{frame}
	\titlepage
\end{frame}

\begin{frame}
	Un environnement \texttt{frame} pour chaque \emph{diapositive}.
	\visible<2>{Chaque diapo pouvant contenir plusieurs \emph{couches}.}
\end{frame}


\begin{frame}{On peut mettre un titre : Sommaire}
	\tableofcontents
\end{frame}

\section{Section 1}
\begin{frame}{La section 1 commence}
	blabla
\end{frame}

\begin{frame}
	Un \textbf<2,3>{texte} en gras. 
	\visible<3>{Un texte visible sur la 3\ieme{} couche}
\end{frame}

\begin{frame}{Titre (facultatif)} 
\framesubtitle{Sous titre (facultatif aussi)}
	\begin{block}{Remarque}
	Un bloc
	\end{block}
	
	\begin{alertblock}{Proposition}
	Un bloc alerte
	\end{alertblock}
	
	\begin{exampleblock}<2>{Exemple}
	Un bloc exemple qui est visible sur la 2\ieme{} couche : $f(x)=2x$.
	\end{exampleblock}
\end{frame}

\section{Section 2}
\begin{frame}{La section 2 commence}
\begin{itemize}
	\item<1-> On peut cliquer sur les titres de la barre de gauche pour naviguer dans les sections du pdf (essayez !).
	\item<2-> On peut changer le ``look'' du beamer, en changeant de thème. Retournez dans le fichier source et compilez avec les autres thèmes proposés (il existe énormément de thèmes; seuls trois sont proposés dans le source).
\end{itemize}

\end{frame}

\end{document}