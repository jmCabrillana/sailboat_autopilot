\documentclass[12pt,a4paper]{article}
\usepackage{natbib}         % Pour la bibliographie
\usepackage{url}            % Pour citer les adresses web
\usepackage[T1]{fontenc}    % Encodage des accents
\usepackage[utf8]{inputenc} % Lui aussi
\usepackage[frenchb]{babel} % Pour la traduction française
\usepackage{numprint}       % Histoire que les chiffres soient bien

\usepackage{amsmath}        % La base pour les maths
\usepackage{mathrsfs}       % Quelques symboles supplémentaires
\usepackage{amssymb}        % encore des symboles.
\usepackage{amsfonts}       % Des fontes, eg pour \mathbb.

\usepackage[svgnames]{xcolor} % De la couleur
\usepackage{geometry}       % Gérer correctement la taille

%%% Si jamais vous voulez changer de police: décommentez les trois 
%\usepackage{tgpagella}
%\usepackage{tgadventor}
%\usepackage{inconsolata}

%%% Pour L'utilisation de Python
\usepackage{pythontex}

\usepackage{graphicx} % inclusion des graphiques
\usepackage{wrapfig}  % Dessins dans le texte.

\usepackage{tikz}     % Un package pour les dessins (utilisé pour l'environnement {code})
\usepackage[framemethod=TikZ]{mdframed}
% Un environnement pour bien présenter le code informatique
\newenvironment{code}{%
\begin{mdframed}[linecolor=Green,innerrightmargin=30pt,innerleftmargin=30pt,
backgroundcolor=Black!5,
skipabove=10pt,skipbelow=10pt,roundcorner=5pt,
splitbottomskip=6pt,splittopskip=12pt]
}{%
\end{mdframed}
}

% Mettez votre titre et votre nom ci-après
\title{Rapport de TIPE \\
Avance du Périhélie de Mercure}
\author{JJ Fleck, PCSI\oldstylenums{1}, \oldstylenums{2015}-\oldstylenums{2016}}
%% À décommenter si vous ne voulez pas que la date apparaisse explicitement
%\date{}

% Un raccourci pour composer les unités correctement (en droit)
% Exemple: $v = 10\U{m.s^{-1}}$
\newcommand{\U}[1]{~\mathrm{#1}}

% Pour discuter avec le prof dans le document: le premier argument est 
% le nom de celui qui fait la remarque et le second la remarque 
% proprement dite: \question{jj}{Que voulez-vous dire par là ?}
% \reponse{Droopy}{I'm very happy...}
\usepackage{todonotes}
\newcommand{\question}[2]{\todo[inline,author=#1]{#2}}
\newcommand{\reponse}[2]{\todo[inline,color=green,author=#1]{#2}}

% Les guillemets \ofg{par exemple}
\newcommand{\ofg}[1]{\og{}#1\fg{}}
% Le d des dérivées doit être droit: \frac{\dd x}{\dd t}
\newcommand{\dd}{\text{d}}


% NB: le script TeXcount permet de compter les mots utilisés dans chaque section d'un document LaTeX. Vous en trouverez une version en ligne à l'adresse
% http://app.uio.no/ifi/texcount/online.php
% Il suffit d'y copier l'ensemble du présent document (via Ctrl-A/Ctrl-C puis Ctrl-V dans la fenêtre idoine) pour obtenir le récapitulatif tout en bas de la page qui s'ouvre alors.

% Pour récupérer les bonnes entrées bibliographiques, je vous conseille l'usage de scholar.google.fr pour les recherches
% et la récupération des entrée BibTeX comme décrit dans cette vidéo: https://www.youtube.com/watch?v=X-9T2Oaj-5A

% Quelques macros utiles
% La dérivée temporelle, tellement courante en physique, avec les d droits
\newcommand{\ddt}[1]{\frac{\dd #1}{\dd t}}
% Des parenthèses, crochets et accolades qui s'adaptent automatiquement à la taille de ce qu'il y a dedans
\newcommand{\pa}[1]{\left(#1\right)}
\newcommand{\pac}[1]{\left[#1\right]}
\newcommand{\paa}[1]{\left\{#1\right\}}
% Un raccourci pour écrire une constante
\newcommand{\cte}{\text{C}^{\text{te}}}
% Pour faire des indices en mode texte (comme les énergie potentielles)
\newcommand{\e}[1]{_{\text{#1}}}
% Le produit vectoriel a un nom bizarre:
\newcommand{\vectoriel}{\wedge}

\begin{document}

\maketitle

\begin{abstract}
%   Find something meaningful to say in english in order to describe the present report...
    
    Relativity is one of the most beautiful theories we can come across in physics. It has some magic in it as it is mainly based on mere \ofg{thought experiments}. Nevertheless, in Phycics, it is experimental evidence that decides whether a theory is worth developping and the advance of Mercury's perihelion was one of the most exciting moments of truth Einstein faced with its \ofg{General Relativity} theory.
    
    In this work, we tried to understand how some glimpses of relativity could shed light on one of the late \textsc{xix}$^{\text{th}}$ century mysteries.
\end{abstract}

\section{Introduction à la dynamique relativiste : relation
fondamentale et conservation de l'énergie}

On peut s'interroger sur la forme que prendra la généralisation
relativiste de la relation fondamentale de la dynamique newtonienne :

\begin{equation}
    \ddt{\vec{p}} = \sum \vec{F} 
    \qquad \text{avec} \qquad 
    \vec{p} = m\vec{v}
\end{equation}

    On retrouve en fait la même formulation, mais avec cette fois

\begin{equation}
    \vec{p} = \gamma \, m \vec{v} 
    \qquad\text{où}\qquad
    \gamma=\frac{1}{\sqrt{1 - \frac{v^2}{c^2}}}
\end{equation}

\noindent dont découle une nouvelle forme de la dérivée du vecteur
quantité de mouvement :



\begin{equation}
    \ddt{\vec{p}} = 
    \gamma\, m \ddt{\vec{v}} + \ddt{\gamma}\, m \vec{v}
\end{equation}

On remarque qu'ainsi, le vecteur accélération du point matériel
auquel on applique la relation n'est plus nécessairement colinéaire
à la résultante des forces appliquées à ce point.

    On peut aussi démontrer un nouveau théorème de l'énergie : 

\begin{equation}
    \ddt{}\!\pa{E_0 + E\e{c} + E\e{p}} = \mathcal{P}'
\end{equation}

\noindent où $\mathcal{P}'$ représente la puissance des forces non
conservatives, $E\e{p}$ l'ensemble des énergies potentielles des forces
conservatives, $E\e{c} = (\gamma - 1)\, m\, c^2$ une nouvelle définition de
l'énergie cinétique et $E_0=m\, c^2$ l'énergie propre de l'objet
directement proportionnelle à sa masse.
    
    Pour ce qui est de la nouvelle définition de l'énergie
cinétique, on peut remarquer qu'en développant $\gamma - 1$ au premier
ordre en $\frac{v^2}{c^2}$, on retrouve la formule bien connue : $E\e{c} =
\frac{m\, v^2}{2}$.

    Dans le cas (courant en astronomie) où la puissance des forces
non conservatives est négligeable, on obtient la relation :
    
\begin{equation}
    E_0 + E\e{c} + E\e{p} = E = \cte
\end{equation}

    \section{Application à une planète $A$ soumise à la force
gravitationnelle d'une étoile placée en $O$}

La relation fondamentale, pour une planète de masse $m$, s'écrit 

\begin{equation}
    \ddt{\vec{p}} = -\frac{G M m}{r^3}\, \vec{r}
\end{equation}

\noindent
avec $\vec{r} = \overrightarrow{OA}$ et $M$ masse de l'étoile exerçant la
force.

    On peut montrer dans une telle situation (voir équation~\ref{moment:cinetique} dans les annexes, page~\pageref{moment:cinetique})
que le moment cinétique de la planète est constant et que le mouvement
est donc plan. On obtient, avec $\vec{L} = \vec{r}\vectoriel\vec{p}$
où $\vectoriel$ est le produit vectoriel,

\begin{equation}
    L = \gamma\, m \, r^2 \, \ddt{\theta} = \cte
\end{equation}

    En utilisant la conservation de l'énergie qui permet d'atteindre
$\gamma$ et son expression littérale que l'on va exprimer en fonction de
la nouvelle variable $u = 1/r$ et de ses dérivées en fonction de
$\theta$ (cf équation~\ref{binet}, page~\pageref{binet}), on arrive à l'équation 

\begin{equation}
    \boxed{
    \frac{\dd^2 u}{\dd \theta^2} + u = \frac{G\, M\, m\, E}{L^2\, c^2}
        + \frac{\pa{G\,M\, m}^2}{L^2\, c^2}\, u
    }
\end{equation}


\begin{flalign}
\text{Qui se réduit à} & &
    \frac{\dd u}{\dd \theta} + B^2\, u &= A 
    \quad \text{avec} \quad 
    B^2  = 1 - \pa{\frac{G M m}{L\, c}}^{\!\!2} &
\end{flalign}

\begin{flalign}
\text{On obtient donc} & &
    r &= \frac{p}{1+e\cos{\pac{B\pa{\theta - \theta_0}}}} &
\end{flalign}

\begin{flalign}
\text{où}   & &
    p&=\frac{L^2\, c^2 - (G M m)^2}{G M m\, E}  
    \qquad\text{et}\qquad
    e=C\, \frac{B^2}{A} 
    &   %\\[4mm]
%\text{et}  & &
%   e&=C\, \frac{B^2}{A} &
\end{flalign}

\noindent avec $C$ constante d'intégration.
Si \mbox{$e<1$}, comme c'est le cas pour Mercure, on observe sensiblement
une ellipse avec la particularité de ne pas se refermer sur elle-même.
Entre deux passages au point le plus proche du soleil de son orbite
\mbox{($\cos\pac{B(\theta-\theta_0)}=1$)}, son vecteur position aura
tourné d'un angle \mbox{$2\pi + \delta$} avec l'expression suivante de
$\delta$ 

\begin{equation}
    \boxed{
    \delta = 2\pi\pa{\frac{1}{B} - 1}
    \approx \pi \pa{\frac{G\, M\, m}{L\, c}}^{\!\!2} 
    = \pi \, \frac{G M}{p\, c^2}
    = \pi\, \frac{G M}{a\, c^2\, \pa{1-e^2}}
    }
\end{equation}

\begin{figure}[ht]
\begin{center}
\includegraphics[width=10cm]{Fig01}      
\end{center}
\caption{Illustration du phénomène de précession du périhélie}
\end{figure}


Malheureusement, les valeurs numériques pour Mercure ne donnent qu'une
précession de $7''$ d'angle par siècle et non pas les $43''$ d'angle par
siècle escomptées : il~manque le facteur~6 que l'on trouvera dans la
formule donnée par la relativité générale. Il est néanmoins
intéressant de constater que la relativité, même restreinte, donne
déjà une orbite qui précesse là où Newton ne voit qu'une ellipse
stable.

\subsection{Le point de vue de la relativité générale}

Le développement de la relativité générale provient de la remarque
d'Einstein portant sur l'identité (à priori non triviale) entre masse
grave et masse inerte. La masse grave est celle que l'on prend en compte
lorsque l'on décrit l'interaction gravitationnelle entre 2 corps. La
masse inerte, en revanche, est celle que l'on utilise pour écrire la
relation fondamentale de la dynamique :
    
\begin{equation}
    m\, \vec{a} = \sum \vec{F}
\end{equation}

    Plusieurs expériences ont tendu à prouver au fil des siècles
l'identité de ces deux notions (ou au moins un facteur de
proportionnalité entre elles). Partant de cette constatation, Einstein a
posé l'équivalence entre le champ gravitationnel et l'accélération
: aucune expérience ne peut révéler de différence entre un
laboratoire local accéléré et son analogue sur la terre soumis à
la gravitation.

Il en découle que l'interprétation de la gravitation en terme
d'interaction résulterait de la place privilégiée accordée aux
référentiels galiléens alors que rien ne la justifie à priori dans
la théories newtonienne ou celle de la relativité restreinte. Einstein
en est donc arrivé à développer un espace-temps qui contiendrait
déjà en lui l'interaction gravitationnelle, tout comme il apparaît des
accélérations d'entraînement et de Coriolis dans un référentiel
non galiléen en mécanique classique.

\begin{figure}[ht]

 \begin{center}
\includegraphics[width = 12cm]{Fig02}
\caption{Comparaison des deux visions}
      
\end{center}
\end{figure}

\subsection{Particule au voisinage d'une masse à symétrie sphérique}

On peut toujours appliquer la relation fondamentale de la dynamique
généralisée, mais la différence introduite par la relativité
générale prend forme dans l'expression de la vitesse qui ne s'écrit
plus en coordonnée polaire :

\begin{equation}
    \vec{v} = \ddt{\vec{r}} = \ddt{r}\, \vec{e_r}
            + r\,\ddt{\theta}\, \vec{e_{\theta}}
\end{equation}

\noindent mais, avec la nouvelle métrique où $r_0 = \frac{G M}{c^2}$,

\begin{equation}
    \vec{v} = \frac{1}{1-2\, \frac{r_0}{r}}\, \ddt{r}\, \vec{e_r}
    + \frac{r}{\sqrt{1-2\, \frac{r_0}{r}}} \, \ddt{\theta}\, \vec{e_{\theta}}
\end{equation}

\noindent   De même, l'énergie s'exprime à présent 
 
\begin{equation}
    E = \gamma\, m\, c^2\, \sqrt{1-2\, \frac{r_0}{r}} = \cte 
\end{equation}


\noindent pour le cas céleste. On montre comme précédemment que le moment cinétique $\vec{L}$ de la planète est constant et vaut :

\begin{equation}
    \vec{L} = \frac{\gamma\, m\, r^2}{\sqrt{1 - 2\,\frac{r_0}{r}}}\,
            \ddt{\theta}\, \vec{e_z}
\end{equation}

Après substitution de $v^2$ en fonction de $\gamma$ et introduction de
la variable $u = 1/r$ (cf annexe 1), on obtient l'équation
différentielle suivante :
            
\begin{equation}
    \boxed{
    \frac{\dd^2 u}{\dd \theta^2} + u 
    = \frac{G M m^2}{L^2} + \frac{3\, G M}{c^2}\, u^2
    }
\end{equation}

Vis à vis de l'équation Newtonienne, il apparaît un terme
quadratique d'amplitude très faible qui induira la précession
observée de Mercure de $43''$ d'angle par siècle. La résolution de
cette équation avec Maple (cf annexe 2) donne la formule suivante pour
l'avance sur une période :

\begin{equation}
    \boxed{
    \delta = 6\, \pi \, \frac{G M}{a\, c^2\pa{1-e^2}}
    }
\end{equation}

    Et l'application numérique correspondante pour un siècle donne
bien les $43''$ manquantes.

\section{Conclusion}

    Au milieu du siècle, d'autres théories se sont développées 
avec l'ambition de surpasser la relativité dans son application à
l'avancée du périhélie de Mercure. Plus exactement, par
l'intervention de nouvelles notions dans le cadre de la mécanique
classique, elles essayèrent de rajouter quelques secondes d'angle à la
contribution newtonienne, ce qui aurait pour effet de rendre la correction
relativiste trop importante et invaliderait donc la théorie.

La prise en compte de la non sphéricité parfaite du soleil par les
détracteurs de la relativité déboucha notamment sur la constatation
que, passée une certaine valeur de cette non-sphéricité, on
observerait effectivement une précession d'origine non-relativiste.
Malheureusement pour cette théorie, les dernières mesures effectuées
dans ce domaine penchent pour une correction minime due à cet effet et
la prédiction relativiste s'accorde encore aujourd'hui merveilleusement
bien aux incertitudes d'observation.

%\newpage


\begin{appendix}

\section{Calculs}


\subsection{Le point de vue de la relativité restreinte}

Posons les notations suivantes:

\begin{flalign}
\text{Vecteur position de la particule}
&&  \vec{r} &= \overrightarrow{OA} = r\, \vec{e_r}  &\\[4mm]
\text{Vecteur vitesse de la particule}
&&  \vec{v} &= \ddt{\vec{r}} = \ddt{r}\,\vec{e_r}
                + r\, \ddt{\theta}\, \vec{e_\theta} &   \\[4mm]
\text{Quantité de mouvement de la particule}
&&  \vec{p} &= \gamma\, m\, \vec{v} &   \\[4mm]
\text{Moment cinétique de la particule $A$ en $O$}
&&  \vec{L} &= \vec{r}\vectoriel\vec{p} &
\end{flalign}


\noindent Remarquons que, comme $\ddt{\vec{r}}\vectoriel\vec{p} =
\vec{v}\vectoriel\gamma\, m\, \vec{v} = \vec{0}$, on a

\begin{equation}
    \ddt{\vec{L}} = \ddt{\vec{r}}\vectoriel\vec{p} +
    \vec{r}\vectoriel\ddt{\vec{p}}
    = \vec{r}\vectoriel\ddt{\vec{p}}
\end{equation}

\noindent  Appliquons la relation fondamentale de la dynamique à la planète 

\begin{equation}
    \ddt{\vec{p}} = -\frac{G M m}{r^3}\, \vec{r}
\end{equation}


\begin{flalign}
\text{d'où} &&
    \ddt{\vec{L}} &= \vec{r}\vectoriel\ddt{\vec{p}} = \vec{0}   & \label{moment:cinetique}
\end{flalign}

\noindent On a \emph{conservation du moment cinétique}, le mouvement est
donc plan. Par conséquent

\begin{equation}
    \vec{L} = L\, \vec{e_z} \qquad \text{avec} \qquad 
    L = \gamma\, m\,r^2\, \ddt{\theta} = \cte
\end{equation}

\noindent On a aussi \emph{conservation de l'énergie}, avec 
$E\e{c} = (\gamma- 1)\, m\, c^2$ et $E_0 = m\, c^2$,

\begin{equation}
    E_0 + E\e{c} + E\e{p} = \gamma\, m\, c^2 - \frac{G M m}{r} = E = \cte
\end{equation}

\noindent On introduit la variable $u=1/r$ et on en déduit une expression
de $\gamma$

\begin{equation}
    \gamma = \frac{E + G M m\, u}{m\, c^2}
\end{equation}
      
\noindent  On va extraire de la vitesse une autre expression de $\gamma$

\begin{flalign}
\text{Comme}    &&  
    \gamma &= \frac{1}{\sqrt{1-\frac{v^2}{c^2}}} &
\end{flalign}

\begin{flalign}
\text{On en déduit} &&  
    \frac{1}{\gamma^2} &= 1 - \frac{v^2}{c^2}   &
\end{flalign}
    
\begin{flalign}
\text{Or}   &&
    v^2 &= \pa{\ddt{r}}^2 + \pa{r\, \ddt{\theta}}^2 &
\end{flalign}

\noindent Exprimons $\ddt{r}$ et $r\, \ddt{\theta}$ en fonction de la
nouvelle variable $u=\frac{1}{r}$

\begin{align}
    \ddt{r} &= \frac{\dd r}{\dd \theta}\, \ddt{\theta} 
        = \frac{L}{\gamma\, m\,
        r^2}\, \frac{\dd r}{\dd \theta}
    = -\frac{L}{\gamma\, m}\, \frac{\dd u}{\dd \theta} \\[4mm]
    r\, \ddt{\theta} &= r\, \frac{L}{\gamma\, m\, r} 
    = \frac{L}{\gamma \, m}\, u \label{binet}
\end{align}


\noindent D'où l'expresion de $\gamma^2$, au premier ordre en
$\frac{v^2}{c^2}$,

\begin{equation}
    \gamma^2 = 1 + \frac{v^2}{c^2} 
    = 1 + \pa{\frac{L}{m\, c}}^{\!\!2} \, \pac{u^2 +
    \pa{\frac{\dd u}{\dd \theta}}^{\!\!2}}
\end{equation}


\begin{flalign}
\text{Or}   &&  
    \gamma^2 &= \pa{\frac{E + G\, M\, m\, u}{m\, c^2}}^{\!\!2}  &
\end{flalign}

\begin{flalign}
\text{On en déduit l'équation}  &&
    u^2 + \pa{\frac{\dd u}{\dd \theta}}^2 
    &= \pa{\frac{m\, c}{L}}^{\!\!2} \pac{\pa{\frac{E + G
    M m\, u}{m\, c^2}}^{\!\!2} - 1} &
\end{flalign}

\noindent En dérivant cette équation par rapport à $\theta$, en
simplifiant par $2\, \frac{\dd u}{\dd \theta}$, on obtient
    
\begin{equation}
    \frac{\dd^2 u}{\dd \theta^2} + u 
    = \pa{\frac{m\, c}{L}}^{\!\!2} 
    \pa{\frac{E + G M m\, u}{m\, c^2}}^{\!\!2} \frac{G M m}{mc^2}
\end{equation}

\noindent En redéveloppant les différents termes, on obtient

\begin{equation}
    \boxed{
    \frac{\dd^2u}{\dd\theta^2} + u 
    = \frac{G M m\, E}{L^2\, c^2} 
    + \frac{\pa{G M m}^2}{L^2\, c^2}\, u
    }
\end{equation}
    
\begin{flalign}
\text{Qui se réduit à}  &&
    \frac{\dd^2u}{\dd\theta^2} + B^2\, u &= A
    \quad\text{avec}\quad 
    B^2= 1 - \pa{\frac{G M m}{L\, c}}^{\!\!2} &
\end{flalign}

\begin{flalign}
\text{On obtient donc}&&
    r &= \frac{p}{1+e\cos{\pac{B\pa{\theta - \theta_0}}}} &
\end{flalign}
 
\begin{flalign}
\text{où}   &&
    p&=\frac{L^2\, c^2 - (G M m)^2}{G M m E}    &
\end{flalign}

... Vous avez compris, je m'arrête là

\section{Code Python}

Cette section est juste là pour montrer comment on peut rajouter du code Python depuis l'intérieur du code \LaTeX.

% L'environnement {pyverbatim} se contente de recopier le code sans l'exécuter. 
% Pour l'exécuter en plus, il faut utiliser {pyblock}, mais malheureusement Numpy, Scipy et Matplotlib ne sont pas encore accessibles via Overleaf
\begin{code}
\begin{pyverbatim}[][numbers=left]
import numpy as np
import scipy as sp
import scipy.integrate
import matplotlib.pyplot as plt

# Mettez ici des choses intéressantes sachant que PythonTeX
# n'est malheureusement pas encore capable (sur overleaf) de 
# fonctionner avec Numpy, Scipy et matplotlib, donc mieux vaut
# tester le programme dans Pyzo et le recopier ici quand on est 
# sûr du résultat.
\end{pyverbatim}
\end{code}

\end{appendix}
\end{document}